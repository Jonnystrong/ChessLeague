\documentclass{article}

% Language setting
% Replace `english' with e.g. `spanish' to change the document language
\usepackage[english]{babel}

% Set page size and margins
% Replace `letterpaper' with`a4paper' for UK/EU standard size
\usepackage[a4paper,top=2cm,bottom=2cm,left=3cm,right=3cm,marginparwidth=1.75cm]{geometry}

% Useful packages
\usepackage{amsmath}
\usepackage{graphicx}
\usepackage[colorlinks=true, allcolors=blue]{hyperref}

\title{Chess League}
\author{Zeyecx,Jonny}

\begin{document}
	\maketitle
	 
\begin{abstract}
	The idea is to develop a chess tournament in which teams of different streamers play against each other.
	This will first be done in a group system which will later be replaced by a knockout system. 
\end{abstract}


\section{Team structure}
The teams consist of 6 players per team. These must be a active viewer of the respective streamer. 
The game is played in different rating groups (1000, 1400, 1800, 2000, 2200, Streamer).

\section{Rating groups}

The rating groups are:
\begin{itemize}
	\item 1000
	\item 1400 
	\item 1600
	\item 1800
	\item 2000
	\item 2200
\end{itemize}
If there is a tie, the two streamers play against each other. If they do not want to do this, they can appoint a sub to play in their place.

\section{Time managment}
The streamers have 2 weeks to find a date in the group phase. The dates for the knockout phase will be announced separately.
The streamers have to register for the event in advance. Then they get 1 week to set up their team.
Then the event starts.
All matches must be streamed.
However, since we have no registrations, the schedules will be announced elsewhere.

\section{Appendix}
This is the basic version. Further changes will follow
	
\end{document}