
\documentclass[preprint,12pt]{elsarticle}

\usepackage{amssymb}

\usepackage{lineno}

\journal{Chess League - Zeyecx}
\begin{document}
	
\begin{frontmatter}
		
		
\title{Chess League}
		
\author{Zeyecx, Donbur, Jonny}
		
\address{German Paper}
	 
\begin{abstract}
	Die Idee ist es, ein Schachturnier zu entwickeln, in dem Teams verschiedener Streamer gegeneinander spielen.
	Dies wird zunächst in einem Gruppensystem geschehen, das später durch ein K.O.-System ersetzt wird. 
\end{abstract}
\end{frontmatter}
\linenumbers
\section{Teams}
Die Teams bestehen aus 6 Spielern pro Team. Diese müssen aktive Zuschauer des jeweiligen Streamers sein. 
Das Spiel wird in verschiedenen Bewertungsgruppen gespielt

\section{Ratinggruppen}

Die Ratinggruppen sind:
\begin{itemize}
	\item 1000
	\item 1400 
	\item 1600
	\item 1800
	\item 2000
	\item 2200
\end{itemize}
Bei Gleichstand spielen die beiden Streamer gegeneinander. Wenn sie dies nicht tun möchten, können sie einen Ersatzspieler benennen, der an ihrer Stelle spielt.

\section{Zeitplan}
Die Streamer haben 2 Wochen Zeit, einen Termin in der Gruppenphase zu finden. Die Termine für die K.O.-Phase werden separat bekannt gegeben.
Die Streamer müssen sich im Vorfeld für das Event anmelden. Dann haben sie 1 Woche Zeit, ihr Team aufzustellen.
Dann beginnt das Event.
Alle Spiele müssen gestreamt werden.
Da wir aber keine Anmeldungen haben, werden diese an anderer Stelle bekannt gegeben.

\section{Anhang}
Dies ist eine Rohfassung. Wir behalten uns vor, weitere Änderungen vorzunehmen.





\bibliographystyle{./template/model1-num-names}

	
\end{document}
